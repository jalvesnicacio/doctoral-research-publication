\documentclass[a4paper,man,hidelinks,floatsintext,x11names]{apa7}
% This LaTeX output is designed to use APA7 style and to run on local TexLive-installation (use pdflatex) as well as on web interfaces (e.g., overleaf.com).
% To use APA6 style change apa7 to apa6 in the first line (\documentclass), comment or remove the \addORCIDlink line, and change the order of \caption and \label lines for the figures.
% If you prefer postponing your figures and table until after the reference list, instead of having them within the body of the text, please remove the ",floatsintext" from the documentclass options.
% Further information on these styles can be found here: https://www.ctan.org/pkg/apa7 and here: https://www.ctan.org/pkg/apa6

\usepackage[british]{babel}
\usepackage[utf8]{inputenc}
\usepackage{amsmath}
\usepackage{graphicx}
\usepackage[export]{adjustbox}
\usepackage{csquotes}
\usepackage{soul}
%\usepackage[style=apa,sortcites=true,sorting=nyt,backend=biber]{biblatex}
%\DeclareLanguageMapping{british}{british-apa}
%\addbibresource{article.bib}

\title{APA-Style Manuscript with jamovi Results}
\shorttitle{jamovi Results}
\author{Full Name}
\leftheader{Last name}
\affiliation{Your Affilitation}
% addORCIDlink is only available from apa7
\authornote{\addORCIDlink{Your Name}{0000-0000-0000-0000}\\
More detailed information about how to contact you.\\
Can continue over several lines.
}

\abstract{Your abstract here.}
\keywords{keyword 1, keyword 2}

\begin{document}
%\maketitle

% Your introduction starts here.

%\section{Methods}
% Feel free to adjust the subsections below.

%\subsection{Participants}
% Your participants description goes here.

%\subsection{Materials}
% Your description of the experimental materials goes here.

%\subsection{Procedure}
% Your description of the experimental procedures goes here.

%\subsection{Statistical Analyses}
%The description of the statistical procedures that were used to analyse your data goes here.


% =========================================================================================================

\section{Resultados}

           
      
        
      
    
      
      
    
            
\section{Descriptives}

           
      
        
      
    
      
      
    
      
        
      
    
      
      
    
            
\section{L'âge des participants}

           
      
        
      
    
      
      
    %\begin{figure}[htbp]
%\caption{PLACEHOLDER}
%\label{fig:Figure_1}
% (the following arrangement follows APA7; if you want to use APA6, the caption- and label-lines have to be moved to after the includegraphics-line)
% the figure file could not be exported, the LaTeX command below including that figure was therefore commented out
%\centering
%\includegraphics[width=\columnwidth]{figure_1.png}
%\end{figure}

      
        
      
    
      
      
    
      
      
    
      
        
      
    
      
      
    
            
\section{Lesquelles de ces notations utilisez-vous ?}

           
      
        
      
    
\begin{table}[!htbp]
\caption{Multi Response}
\label{tab:Table_1}
\begin{adjustbox}{max size={\columnwidth}{\textheight}}
\centering
\begin{tabular}{lrrr}
\toprule
Option                            & Frequency & Percentage of responses & Percentage of cases \\
\midrule
Use UML                           &        28 &                   71.79 &               82.35 \\
Use ArchiMate                     &         0 &                    0.00 &                0.00 \\
Use OMG Model Driven Architecture &         5 &                   12.82 &               14.71 \\
Use SysML                         &         0 &                    0.00 &                0.00 \\
Use None                          &         3 &                    7.69 &                8.82 \\
Use Other (ER model)              &         2 &                    5.13 &                5.88 \\
Use Other (Flowchart)             &         1 &                    2.56 &                2.94 \\
Total:                            &        39 &                  100.00 &              114.71 \\
\bottomrule
\end{tabular}
\end{adjustbox}
\begin{tablenotes}[para,flushleft] {
\small
\textit{Note.}~These responses were provided by 34 cases.. \\
}
\end{tablenotes}
\end{table}


      
        
      
    
      
      
    
      
        
      
    
      
      
    
            
\section{Quel poste occupez-vous (ou occupiez-vous)?}

           
      
        
      
    
\begin{table}[!htbp]
\caption{Multi Response}
\label{tab:Table_2}
\begin{adjustbox}{max size={\columnwidth}{\textheight}}
\centering
\begin{tabular}{lrrr}
\toprule
Option                 & Frequency & Percentage of responses & Percentage of cases \\
\midrule
Software architect     &         8 &                    9.30 &               23.53 \\
Fullstack developer    &        14 &                   16.28 &               41.18 \\
Backend developer      &        16 &                   18.60 &               47.06 \\
System administrator   &         4 &                    4.65 &               11.76 \\
Frontend developer     &        10 &                   11.63 &               29.41 \\
Mobile developer       &         6 &                    6.98 &               17.65 \\
QA Test developer      &         5 &                    5.81 &               14.71 \\
DevOps developer       &         3 &                    3.49 &                8.82 \\
Database administrator &         5 &                    5.81 &               14.71 \\
Desktop developer      &         3 &                    3.49 &                8.82 \\
Data scientist         &         2 &                    2.33 &                5.88 \\
Embedded developer     &         2 &                    2.33 &                5.88 \\
Product manager        &         4 &                    4.65 &               11.76 \\
Game developer         &         1 &                    1.16 &                2.94 \\
Designer               &         3 &                    3.49 &                8.82 \\
Total:                 &        86 &                  100.00 &              252.94 \\
\bottomrule
\end{tabular}
\end{adjustbox}
\begin{tablenotes}[para,flushleft] {
\small
\textit{Note.}~These responses were provided by 34 cases.. \\
}
\end{tablenotes}
\end{table}


      
        
      
    
      
      
    
      
        
      
    
      
      
    
            
\section{Statistiques descriptives}

           
      
        
      
    
\begin{table}[!htbp]
\caption{Statistiques descriptives}
\label{tab:Table_3}
\begin{adjustbox}{max size={\columnwidth}{\textheight}}
\centering
\begin{tabular}{lr}
\toprule
~          & Âge \\
\midrule
N          &  34 \\
Manquants  &   0 \\
Moyenne    &   ~ \\
Médiane    &   ~ \\
Ecart-type &   ~ \\
Minimum    &   ~ \\
Maximum    &   ~ \\
\bottomrule
\end{tabular}
\end{adjustbox}
\begin{tablenotes}[para,flushleft] {
\small
}
\end{tablenotes}
\end{table}


      
        
      
    
\subsection{Fréquences}

      
        
      
    
\begin{table}[!htbp]
\caption{Fréquences de Âge}
\label{tab:Table_4}
\begin{adjustbox}{max size={\columnwidth}{\textheight}}
\centering
\begin{tabular}{lrrr}
\toprule
Âge            & Quantités & \% du Total & \% cumulés \\
\midrule
18-29 ans      &        12 &      35.3\% &     35.3\% \\
30-39 ans      &        14 &      41.2\% &     76.5\% \\
40-49 ans      &         6 &      17.6\% &     94.1\% \\
50-59 ans      &         1 &       2.9\% &     97.1\% \\
60 ans et plus &         1 &       2.9\% &    100.0\% \\
\bottomrule
\end{tabular}
\end{adjustbox}
\begin{tablenotes}[para,flushleft] {
\small
}
\end{tablenotes}
\end{table}


      
        
      
    
      
      
    
      
        
      
    
\subsection{Graphes}

      
        
      
    
\subsubsection{Âge}

      
        
      
    
      
      
    %\begin{figure}[htbp]
%\caption{PLACEHOLDER}
%\label{fig:Figure_2}
% (the following arrangement follows APA7; if you want to use APA6, the caption- and label-lines have to be moved to after the includegraphics-line)
% the figure file could not be exported, the LaTeX command below including that figure was therefore commented out
%\centering
%\includegraphics[width=\columnwidth]{figure_2.png}
%\end{figure}

% =========================================================================================================

% Report your results here and make references to tables (see Table~\ref{tab:Table_1}) or figures (see Figure~\ref{fig:Figure_1}).

%\section{Discussion}
% Your discussion starts here.

%\printbibliography

%\appendix

%\section{Additional tables and figures}

%Your text introducing supplementary tables and figures.

% If required copy tables and figures from the main results here.

\end{document}

